\documentclass[twocolumn,a4paper]{article}

\usepackage[landscape,left=2cm,right=2cm,top=1cm,bottom=1cm]{geometry}

\fboxrule=0.75pt
\setlength{\fboxsep}{0pt}
\setlength{\columnsep}{25pt}

\pagenumbering{gobble}

\usepackage[T1]{fontenc}
\usepackage[utf8]{inputenc}
\usepackage[swedish]{babel}
\usepackage{csquotes}
\usepackage{hyperref}
\usepackage{url}
\usepackage{graphicx}
\usepackage[flushleft]{threeparttable}
\usepackage{booktabs}
\usepackage{amsmath}
\usepackage{amssymb}
\usepackage{caption}

\newcommand{\image}[2][1.0]{
\begin{figure}[ht]
	\centering
	\fbox{\resizebox{#1\columnwidth}{!}{\includegraphics{#2}}}
\end{figure}
}


\begin{document}

%%%%%%%%%%%%%%%%%%%%%%%%%%%%%%%%%%%%%%%%%%%%%%%%%%%%%%%%%%%%%%%%%%%%%%%%%%%%%%%%%%%%%%%%%%%%%%%%%%%%%%%%%%%%%%%%%%%%%%%%%%%%%%%%%%%%%%%%%%%%
\section*{Kap 1}

$p_{\Delta}(t) = \frac{1}{\Delta}$ då $0<t<\Delta$ annars 0
$\int^{\infty}_{-\infty} p_{\Delta}(t) dt = 1$
\newline
\newline
Om $f$ är deriverbar utom i punkterna $a_1, \ldots, a_n$ där den har språng av höjder $b_1,\ldots,b_n$ så är
\begin{equation}
    f'(t) = f'_p(t) + b_1\delta(t-a_1)+\ldots+b_n(t-a_n)
\end{equation}
där $f'_p$ är derivatan som vi kan läsa av från graf med heavside funktion

\section*{Kap 6}
Faltning: $f \ast g(t) = \int_{-\infty}^{\infty} f(t-\tau)*g(\tau) d\tau$

\section*{Kap 8}
\begin{equation}
    tr A = \lambda_1 + \ldots + \lambda_n    
\end{equation}
\begin{equation}
    det A = \lambda_1 * \ldots *\lambda_n    
\end{equation}


\begin{equation}
    p(D) = diag(p(\lambda_1),\ldots,p(\lambda_n))
\end{equation}
p är vårt polynom, t.ex om vi har $e^A$ så blir det, $p(x) = e^x$
\begin{equation}
    p(A) = Sp(D)S^{-1}
\end{equation}
\section*{Kap 9 - Lösa diffekvationer}
Olika sätt att lösa diff ekvationer av matriser
\subsection*{ Laplacetransformation}

\subsection*{Diagonalisering genom variabelbyte}
Om $A$ är en diagonaliserbar matris så har det homogena systemet $\frac{du}{dt} = Au$ den allmänna lösningen
\begin{equation}
    u=C_1e^{\lambda_1t}s_1+\ldots+C_n*e^{\lambda_nt}s_n
\end{equation}
där $\lambda$ är egenvärden till $A$, $s$ är motsvarande egenvektorer och $C$ är godtyckliga konstanter. 
\subsection*{Exponentialmatris}
\begin{equation}
    e^{tA} = Se^{tD}S^{-1} = S diag(e^{\lambda_1t}, \ldots, e^{\lambda_n * T}) S^{-1}
\end{equation}

det homogena systemet $\frac{du}{dt}=Au$ har lösningen $u(t)=e^{tA}u(0)$


\subsection*{Kontrollfrågor}
\begin{equation}
    \delta(t) = \lim_{\Delta \to 0} p_{\Delta}(t)
\end{equation}

\begin{equation}
    \Delta(t)' = \delta(t)
\end{equation}

Har alla funktioner en Laplace transformation? - Funktioner vars integral av laplace inte konvergerar saknar Laplacetransformation.
Ensidig laplace = TODO: Finish this
\newline
\newline
% TODO: Härled derivationsregeln för den ensidiga Laplacetransformationen, 6:00 i video 1
$\delta \ast f(t) = f(t)$


\subsubsection*{System}
\textbf{Vad menas med:} 
\begin{itemize}
    \item Linjärt: $S(\alpha w_1 + \beta w_2) = \alpha S w_1 + \beta S w_2$
    \item Tidsinvariant: Om $S w(t) = y(t)$ så är $S*w(t-a) = y(t-a)$
    \item Stabilt: Varje begränsad insignal w(t) ger upphov till en begränas utsignal y(t). 
    \newline
    Detta kan testas med följande sats: \newline
    Om ett LTI system $S$ har impulssvaret $h(t)$ så är $S$ stabilt om och endast om integralen $\int_{-\infty}{\infty} |h(t) dt$ konvergerar
    Det kan också testas med följande sats: \newline
    Om $H(s) = \frac{Q(s)}{P(s)}$ så är Systemet S stabilt om och endast om $deg Q(s) \leq deg P(s)$ och För varje pol $s_j$ gäller $Re s_j \le 0$.
    \item  Kausalt: Om $S w(t) = y(t)$ och $w(t) = 0$ för $t < t_0$ så gäller att $y(t) = 0$ för $ t < t_0$ 
    \newline
    Detta kan testas med följande sats: \newline
    Ett LTI system är kausalt om och endast om impulssvaret $h(t)$ är en kausal funktion. 
\end{itemize}

\textbf{Bra satser:}\newline
\begin{itemize}
    \item System kan beskrivas som faltningar med en fix funktion $h(x)$. 
    \item Impulssvaret är derivatan av stegsvaret: $h(t) = (S\theta(t))'$
    \item Överföringsfunktionen: $H(s) = \frac{Se^{st}}{e^{st}}$
    \item Frekvensfunktion $H(t)$ ger oss följande:
    \item $S sin(w t) = A(w) sin(wt+ \phi(w))$ med amplitudfunktionen $A(w) = |H(iw)|$ och fasfunktionen $\phi(w) = arg(H(iw))$
    \item För LTI system så är $\mathcal{L} h(t) = H(s)$
    \item $S(e^{st) = H(s)e^{st}$

\end{itemize}

\subsubsection*{Matriser}
Det finns diagonaliserbara matriser med flera av samma egenvärden, t.ex $A = I$ har egenvärden $\lambda_1 = \lambda_2 = 1$ och är diagonal.
\newline
\textbf{Satser:}
\begin{itemize}
    \item $e^{At} = \Sigma^{\infty}_{k=0} \frac{A^k*t^k}{k!}$
    \item Ortogonal matris: $A^{-1}=A^T$, $AA^T=A^TA=I$, % TODO: är del två sann?
    
\end{itemize}



% TODO: Förstår inte helt fråga 21

\end{document}
