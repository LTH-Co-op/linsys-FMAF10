\documentclass[twocolumn,a4paper]{article}

\usepackage[landscape,left=2cm,right=2cm,top=1cm,bottom=1cm]{geometry}

\fboxrule=0.75pt
\setlength{\fboxsep}{0pt}
\setlength{\columnsep}{25pt}

\pagenumbering{gobble}

\usepackage[T1]{fontenc}
\usepackage[utf8]{inputenc}
\usepackage[swedish]{babel}
\usepackage{csquotes}
\usepackage{hyperref}
\usepackage{url}
\usepackage{graphicx}
\usepackage[flushleft]{threeparttable}
\usepackage{booktabs}
\usepackage{amsmath}
\usepackage{amssymb}
\usepackage{caption}

\newcommand{\image}[2][1.0]{
\begin{figure}[ht]
	\centering
	\fbox{\resizebox{#1\columnwidth}{!}{\includegraphics{#2}}}
\end{figure}
}


\begin{document}

%%%%%%%%%%%%%%%%%%%%%%%%%%%%%%%%%%%%%%%%%%%%%%%%%%%%%%%%%%%%%%%%%%%%%%%%%%%%%%%%%%%%%%%%%%%%%%%%%%%%%%%%%%%%%%%%%%%%%%%%%%%%%%%%%%%%%%%%%%%%
\section*{Kap 1}

$p_{\Delta}(t) = \frac{1}{\Delta}$ då $0<t<\Delta$ annars 0
$\int^{\infty}_{-\infty} p_{\Delta}(t) dt = 1$
Om $f$ är deriverbar utom i punkterna $a_1, \ldots, a_n$ där den har språng av höjder $b_1,\ldots,b_n$ så är
$f'(t) = f'_p(t) + b_1\delta(t-a_1)+\ldots+b_n(t-a_n)$
där $f'_p$ är derivatan som vi kan läsa av från graf med heavside funktion

\section*{Kap 6}
Faltning: $f \ast g(t) = \int_{-\infty}^{\infty} f(t-\tau)*g(\tau) d\tau$

\section*{Kap 8}
Produkten av alla egenvärden till matris $A$ är $det A$ \newline
Summan av alla egenvärden till matris $A$ är $tr A$ vilket betyder att vi tar summan av diagonal elementen.

\begin{equation}
    p(D) = diag(p(\lambda_1),\ldots,p(\lambda_n))
\end{equation}
p är vårt polynom, t.ex om vi har $e^A$ så blir det, $p(x) = e^x$
\begin{equation}
    p(A) = Sp(D)S^{-1}
\end{equation}
\section*{Kap 9 - Lösa diffekvationer}
Olika sätt att lösa diff ekvationer av matriser
\subsection*{ Laplacetransformation}

\subsection*{Diagonalisering genom variabelbyte}
Om $A$ är en diagonaliserbar matris så har det homogena systemet $\frac{du}{dt} = Au$ den allmänna lösningen
\begin{equation}
    u=C_1e^{\lambda_1t}s_1+\ldots+C_n*e^{\lambda_nt}s_n
\end{equation}
där $\lambda$ är egenvärden till $A$, $s$ är motsvarande egenvektorer och $C$ är godtyckliga konstanter. 
\subsection*{Exponentialmatris}
\begin{equation}
    e^{tA} = Se^{tD}S^{-1} = S diag(e^{\lambda_1t}, \ldots, e^{\lambda_n * T}) S^{-1}
\end{equation}

det homogena systemet $\frac{du}{dt}=Au$ har lösningen $u(t)=e^{tA}u(0)$


\subsection*{Kontrollfrågor}
\begin{equation}
    \delta(t) = \lim_{\Delta \to 0} p_{\Delta}(t)
\end{equation}

\begin{equation}
    \Delta(t)' = \delta(t)
\end{equation}



\end{document}
